\documentclass[11pt,a4paper]{article}
\usepackage{graphicx}
\usepackage{listings}
\usepackage{hyperref}
\usepackage{ngerman}
\usepackage[latin1]{inputenc}
\pagenumbering{Roman}
\usepackage{multicol}



\begin{document}


\title{PPM - Zeitdokumentation}
\author{Mattias Abfalterer}
\maketitle

\vfill
\pagebreak


\section*{08.10.2014}

\begin{itemize}
\item Besch"aftigen mit GitHub
\item Pflichtenheft besprechen
\end{itemize}



\section*{15.10.2014}

\begin{itemize}
\item Recherche LeapMotion
\item Bestellung Bluetooth-Modul
\item Fertigstellung Pflichtenheft
\item Besch"aftigen mit GitHub-Branches und mergen
\end{itemize}



\section*{22.10.2014}

\begin{itemize}
\item LaTex funktionsf"ahig machen
\item Recherche Kommunikation zwischen Laptop und Arduino
\item Organisatorisches im Team
\end{itemize}



\section*{29.10.2014}

\begin{itemize}
\item Bestellung ben"otigter Hardware
\item Programm Bluetooth Laptop-Arduino
\item Hardware auseinanderbauen (zu Hause)
\item Bestellung zus"atzlicher Hardware (zu Hause)
\end{itemize}



\section*{29.10.2014}

\begin{itemize}
\item Konzept Aussehen des Roboters mit Erweiterung
\end{itemize}


\section*{05.11.2014}

\begin{itemize}
\item Teambesprechung
\item Recherche serielle Komunikation f"ur C\#
\end{itemize}


\section*{12.11.2014}

\begin{itemize}
\item Bluetooth Kommunikation zwischen Laptop und Arduino testen
\item C\# Programm f"ur serielle Kommunikation
\end{itemize}


\section*{19.11.2014}

\begin{itemize}
\item Code f"ur Bluetooth in C\# - Programm, das die Daten von LeapMotion entgegennimmt, einpflegen
\item Ausf"uhrliche Testung der seriellen Kommunikation
\end{itemize}

\subsection*{Noch zu tun:}
\begin{itemize}
\item Arduino-Programm, das die Daten interpretiert und den Motoren "ubergibt
\item Hardware zusammenbauen und testen
\item eventuell noch Zeit freihalten f"ur Troubleshooting
\end{itemize}


\section*{25.11.2014 - Zuhause}
\begin{itemize}
\item Provisorischer Zusammenbau der Hardware
\end{itemize}


\section*{26.11.2014}
\begin{itemize}
\item Troubleshooting f"ur Programm
\item Besprechung der m"oglichen Fehler
\end{itemize}


\begin{multicols}{2}
[\section*{  03.12.2014  }] 


\subsection*{T"atigkeit}
\begin{itemize}
\item Troubleshooting f"ur Programm
\item Fehler: Einlesen der Daten die vom PC kommen
\end{itemize}

\columnbreak

\subsection*{Meilensteine}

\begin{itemize}
\item Eingelesene Daten richtig spalten (10.12.2014)
\item Bewegungsdaten verwerten (17.12.2014)
\end{itemize}

\end{multicols}

\newpage

\begin{multicols}{2}
[\section*{  10.12.2014  }] 


\subsection*{T"atigkeit}
\begin{itemize}
\item String richtig aufgespaltet und in Integer gewandelt
\item Bewegungsdaten verwertet
\end{itemize}

\columnbreak

\subsection*{Meilensteine}

\begin{itemize}
\item Roboter l"oten (17.12.2014)
\item Testen (17.12.2014)
\end{itemize}

\end{multicols}


\begin{multicols}{2}
[\section*{  17.12.2014  }] 


\subsection*{T"atigkeit}
\begin{itemize}
\item Roboter gel"otet
\item Roboter testen
\item Problem: keine LeapMotion
\end{itemize}

\columnbreak

\subsection*{Meilensteine}

\begin{itemize}
\item Test mit LeapMotion (07.01.2015)
\end{itemize}

\end{multicols}



\end{document}